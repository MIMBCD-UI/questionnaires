\documentclass[a4paper, 11pt]{article}
\usepackage{comment} % enables the use of multi-line comments (\ifx \fi) 
\usepackage{lipsum} %This package just generates Lorem Ipsum filler text. 
\usepackage{fullpage} % changes the margin
\usepackage{amssymb}
\usepackage{array,longtable}

\newcounter{question}
\newcounter{row}

\makeatletter
\newtoks\@tabtoks
\newcommand\addtabtoks[1]{\@tabtoks\expandafter{\the\@tabtoks#1}}
\newcommand*\resettabtoks{\@tabtoks{}}
\newcommand*\printtabtoks{\the\@tabtoks}
\makeatother

\newcommand\CheckTable[1]{%
  \setcounter{question}{0}
  \setcounter{row}{0}
  \resettabtoks
  \loop\ifnum\therow<#1\relax
    \stepcounter{row}
    \addtabtoks{ & $\square$ & $\square$ & $\square$ & $\square$ & $\square$ \\}%
  \repeat
  \begin{longtable}{>{\stepcounter{question}}l*{5}{c}}
    \multicolumn{1}{c}{} & 1 & 2 & 3 & 4 & 5 \\        \printtabtoks
  \end{longtable}%
}

\newcommand\CheckTablePref[1]{%
  \setcounter{question}{0}
  \setcounter{row}{0}
  \resettabtoks
  \loop\ifnum\therow<#1\relax
    \stepcounter{row}
    \addtabtoks{ & $\square$ & $\square$ & $\square$ & $\square$ \\}%
  \repeat
  \begin{longtable}{>{\stepcounter{question}}l*{4}{c}}
    \multicolumn{1}{c}{} & 1 & 2 & 3 & 4 \\        \printtabtoks
  \end{longtable}%
}

\begin{document}
%Header-Make sure you update this information!!!!
\noindent
\large\textbf{Traditional Prototype User Observation} \hfill \textbf{Francisco Maria Calisto} \\
\normalsize ISR \& INESC-ID \\
Prof. Dr. Jacinto Nascimento \hfill R\&D Date: 15/07/2017 \\
Prof. Dr. Daniel Gon\c{c}alves \hfill Due Date: 15/07/2017 \\

\section*{Q1: Positive Affect}

\section*{\CheckTable{1}}

\section*{Q2: Negative Affect}

\section*{\CheckTable{1}}

\section*{Q3: Enjoyment}

\section*{\CheckTable{1}}

\section*{Q4: Competence}

\section*{\CheckTable{1}}

\section*{Q5: Autonomy}

\section*{\CheckTable{1}}

\section*{Q6: Relatedness}

\section*{\CheckTable{1}}

\section*{Q7: Immersion}

\section*{\CheckTable{1}}

\section*{Q8: Intuitive Controls}

\section*{\CheckTable{1}}

\section*{Q9: Preferences}

(1=best 4=worst)

\section*{\CheckTablePref{1}}

\end{document}
